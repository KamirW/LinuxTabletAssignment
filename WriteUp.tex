\chapter{Installing Linux onto a Tablet}
\begin{quote}\it
``This chapter contains work done by: ''
\flushright{$-$Ian Bowser & Kamir Walton}
\end{quote}
\label{chapter: Installing Linux onto a Tablet}

# This is the basic overview section... delete this small comment at the end

It is commonly known that tablets are like bigger versions of android phones. And an android phone is just like a small computer. So, following that string of logic, a tablet would then 
just be a bigger computer than an android. With that being the case, is it then possible to put a computer operating system onto a tablet? Well, the short answer is yes, but it is certainly 
no easy feat. This chapter will cover the steps to loading a different operating system onto a tablet, the barriers of such a task, and also the shortcomings and advantages to doing such 
a thing. 

\section{Rooting the Tablet}
In order to override the existing operating system on an Android tablet, it is first necessary to root the device. But what is rooting exactly? Rooting is a process in which 
administrative control is gained over an Android device. This then allows the user to do things such as replacing the firmware of the device; something that is incredibly
useful when it comes to completely changing the operating system of the tablet. The process to doing such a thing is a little lengthy, however. In addition, tablets
have become increasingly hard to root in the recent years as technology has improved its security measures.

\section{Unlocking the Bootloader}
Unlocking the bootloader is just another way of saying "root the device". There are a several steps involved in rooting a device:
\begin{itemize}

\item Install Android SDK
\item Enable USB Debugging and OEM Unlocking
\item Boot Device to Fastboot Mode
\item Unlock Bootloader via Fastboot Command

\end{itemize}

\subsection{Install Android SDK}
Installing the Android SDK Platform Tools is the first step to rooting the device. These are a set of tools that allow a developer to interact with the Android device.
There are two main tools that are included in the SDK Platform Tools that will be useful in rooting the device: adb and fastboot. Adb stands for Android Debug Bridge and is 
a command line tool that allows developers to communicate with the device. It actually allows access to a Unix shell in order to perform such an action. Fastboot is a tool that 
interfaces with the Android platform. It can be used for things like accessing all of a device's partitions or loading the device independently of the operating system 
(which will be useful in this particular case).

****** INSERT A PICTURE OF THE ANDROID SDK PLATFORM TOOLS FOLDER ******

\subsection{Enable USB Debugging and OEM Unlocking}
The second step to rooting the device is enabling USB debugging. This is a setting that can be enabled on an Android device 

\section{Pitfalls to trying to load another operating system}

\section{Advantages & Disadvantages}
